\documentclass[11pt]{article}
\usepackage{graphicx}
\begin{document}
\section{Introduction}
The following document concerns the Link Analysis techniques and is part of the second homework
of the \textit{Web Information Retrieval} course.\\
The goal of this homework is to compute the \textbf{Topic-Specific-PageRank} for the nodes of
a weighted and undirected graph, representing movies and relations among them, to obtain recommendations.\\
Recommendations can be computed biasing the Topic-Specific-PageRank according to users' preferences about
a certain set of movies.

\section{MovieLens 100K Dataset analysis}
\textbf{MovieLens 100K Dataset} is a stable benchmark dataset containing
 100.000 ratings from 943 users in 1682 movies, where each user voted for at least 20 movies.
 Each rating ranges from 1 to 5 according to the degree of user preference.
 \footnote{http://grouplens.org/datasets/movielens/100k/}\\
\subsection{Movie Graph}
This graph is a representation of a dataset called \textit{Movie\_Graph}, whose nodes are movies
with a specific ID, and edges are links between them, each one with a weight.\\
The undirected graph is composed by 1682 nodes and 1966412 edges.
\section{Topic Specific PageRank Implementation}
In the first part we used the \textit{networkx}\footnote{networkx is a Python language software package for the creation
and study of the structure of complex networks} library in order to build the weighted and undirected graph.\\
The algorithm proposed during the second lab session was intended for an unweighted graph;\\
In fact, the porting probability in the PageRank of a specific node j, was the following:\\
\begin{center}
$ r_j =\sum_{i\rightarrow j}(1-\alpha) \frac{PR_{i}^{t}}{\delta_{out}(i)}$
\end{center}
where $i$ is the node adjacent to $j$, $\alpha$ is the teleporting probability, $t$ is a counter for the
iteration and $\delta_{out}(i)$ is the out degree of $i$.\\
The out degree of node $i$ is inserted at denominator because the probability for a random surfer
of going from $i$ to its neighbours is considered uniform.\\
In the homework scope, such probability must be calculated according to edges weights, then
the uniform hypothesis does not hold anymore.\\
The weight w(i,j) of the edge (i,j) must be normalized to get a probability value(in the
range [0,1]):\\
\begin{center}
$weight\_norm(i,j) = \frac{w(i,j)}{\sum_{i\rightarrow k}w(i,k)}$\\
\end{center}
Finally, the porting probability is obtained with the following:\\
\begin{center}
$ r_j =\sum_{i\rightarrow j}(1-\alpha)PR_{i}^{t}$ $weight\_norm(i,j)$\\
\end{center}










\end{document}
