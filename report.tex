\documentclass[11pt]{article}
\usepackage{graphicx}
\usepackage{mathtools}
\begin{document}

\begin{titlepage}
	\begin{center}
		\vspace*{\fill}
		\begin{Huge}
			Homework \#2\\
			Web Information Retrieval
		\end{Huge}
		\newline
		\newline
		Federico Arcangeli \-- 1771113, Alessandro Scir\`{e} \-- 1770594,\\
		Fabrizio Tropeano \--1773456
		\vspace*{\fill}	
	\end{center}
\end{titlepage}
\section{Introduction}
The following document concerns the Link Analysis techniques and is part of the second homework
of the \textit{Web Information Retrieval} course.\\
The goal of this homework is to compute the \textbf{Topic-Specific-PageRank} for the nodes of
a weighted and undirected graph, representing movies and relations among them, to obtain recommendations.\\
Recommendations can be computed biasing the Topic-Specific-PageRank according to users' preferences about
a certain set of movies.

\section{MovieLens 100K Dataset analysis}
\textbf{MovieLens 100K Dataset} is a stable benchmark dataset containing
 100.000 ratings from 943 users for 1682 movies, where each user voted for at least 20 movies.
 Each rating ranges from 1 to 5 according to the degree of user preference.
 \footnote{http://grouplens.org/datasets/movielens/100k/} \\
\subsection{Movie Graph}
This graph is a representation of a dataset called \textit{Movie\--Graph}, whose nodes are movies
with a specific ID, and edges are links between them, each one with a weight.\\
The undirected graph is composed by 1682 nodes and 1966412 edges.
\section{Topic Specific PageRank Implementation}
In the first part we used the \textit{networkx}\footnote{networkx is a Python language software package for the creation
and study of the structure of complex networks} library in order to build the weighted and undirected graph.\\
The algorithm proposed during the second lab session was intended for an unweighted graph; in fact, the porting probability in the PageRank of a specific node j, was the following:\\
\begin{center}
$ r_j =\sum_{i\rightarrow j}(1-\alpha) \frac{PR_{i}^{t}}{\delta_{out}(i)}$
\end{center}
where $i$ is the node adjacent to $j$, $\alpha$ is the teleporting probability, $t$ is a counter for the
iteration and $\delta_{out}(i)$ is the out degree of $i$.\\
The out degree of node $i$ is inserted at denominator because the probability for a random surfer
of going from $i$ to its neighbours is considered uniform.\\
In the homework scope, such probability must be calculated according to edges weights, then
the uniform hypothesis does not hold anymore.\\
The weight $w(i,j)$ of the edge $(i,j)$ must be normalized to get a probability value(in the
range $[0,1]$):\\
\begin{center}
$weight\_norm(i,j) = \frac{w(i,j)}{\sum_{i\rightarrow k}w(i,k)}$\\
\end{center}
Finally, the porting probability is obtained with the following:\\
\begin{center}
$ r_j =\sum_{i\rightarrow j}(1-\alpha)PR_{i}^{t} \cdot weight\_norm(i,j)$\\
\end{center}

\section{Movie recommendation from user's preferences}
In the second part we implemented a method for recommending movies to a given user according to his preferences.
We used the TSP algorithm of the first part applied on \textit{Movie\--Graph} biasing the teleporting probability
distribution with respect to his ratings.\\
The teleporting distribution was defined in the following way: \\
\begin{center}
  $LeakedPR = 1 - \sum_{j}{r_j}$
\end{center}
where $r_j$ is the porting probability value computed for each node $j$.\\
In the first part of the homework the teleporting distribution $P(t)$ was uniform:
given that the surfer is in the node $j$, the probability of teleporting to another node $i$ in the graph is the same
for all $i$. Hence, the teleporting probability $P_{j}(t)$ for a single node $j$ was :
\begin{center}
  $P_{j}(t) = \frac{LeakedPR}{N}$
\end{center}
where $N$ is the number of nodes in the graph.\\
Now the teleporting probability distribution must be modified according to user's ratings. We defined the bias as follows:
\begin{equation}
    bias_{j} =
    \begin{cases*}
      \frac{rating^{u}_{j}}{\sum_{k}{rating^{u}_{k}}} & if $j \in rating^{u}$ \\
      0        & otherwise
    \end{cases*}
  \end{equation}
where $rating^{u}$ is the set of ratings given by user $u$, and $rating^{u}_{j}$ is the rating in this set for a specific
movie $j$. \\
Finally, the biased teleporting probability for a single node $j$ is :
\begin{center}
  $P_{j}(t) = LeakedPR \cdot bias_{j}$
\end{center}
Finally, we filtered out from the results the movies already seen by the user, i.e. already ranked.

\section{\fontsize{13}{14}\selectfont Movie recommendation from input\--independent data}
In this last section we explain the approach we chose to implement a method able to recommend movies without computing
the PageRank at recommendation time. We developed two algorithms to achieve this goal : the former computes the PageRank
for each category of movies and the latter combines the results of the first algorithm into a single output. \\
\subsection{TSP based on categories}
This module applies the PageRank computation considering a different teleporting distribution for each category. The
distribution $P_{C}(t)$ is biased as follows :
\begin{equation}
    bias_{j} =
    \begin{cases*}
      \frac{1}{|C|} & if $j \in C$ \\
      0        & otherwise
    \end{cases*}
  \end{equation}
where C represents a category, i.e. a set of movies.\\
Hence, we obtained five different page rank vectors, one for each category.

\subsection{PageRank aggregation}
This module takes as input the PageRank vectors generated by the offline software and a user preferences vector.\\
This vector is composed by five ratings, one for each category, greater or equal than zero.\\
The output is a set of movies recommended for a given user, sorted by PageRank.\\ 
These values comes from the linear combination of PageRank vectors with the coefficients of the user preferences'
vector.\\ 
\\
$user\_preference\_vector = [\alpha, \beta, \gamma, \delta, \epsilon]$\\
\\
$recommendation\_vector =$ \scalebox{1.25}{%
 $\frac{\alpha \cdot PR_1 + \beta \cdot PR_2 +
 \gamma \cdot PR_3 + \delta \cdot PR_4 + \epsilon \cdot PR_5}{\alpha +\beta + \gamma + \delta + \epsilon}$}\\
 \\
 where $PR_i$ is the TSP for category $i$.\\

\end{document}
